\documentclass[a4paper,11pt,english]{article}

\usepackage[latin1]{inputenc}
\usepackage[T1]{fontenc}
\usepackage{newtxtext} % times
\usepackage{amsmath}
\usepackage[varqu,varl]{inconsolata} % typewriter
\usepackage{microtype}
\usepackage{url}
\urlstyle{same}
\usepackage{xcolor}

\usepackage[pdftex,colorlinks=true,citecolor=blue,
            pagecolor=blue,linkcolor=blue,menucolor=blue,
            urlcolor=blue]{hyperref}
\hypersetup{%
  bookmarksopen=true,
  bookmarksnumbered=true,
  pdftitle={Bayesian data analysis},
  pdfsubject={Reading instructions},
  pdfauthor={Aki Vehtari},
  pdfkeywords={Bayesian probability theory, Bayesian inference, Bayesian data analysis},
  pdfstartview={FitH -32768}
}


% if not draft, smaller printable area makes the paper more readable
\topmargin -4mm
\oddsidemargin 0mm
\textheight 225mm
\textwidth 160mm

%\parskip=\baselineskip
\def\eff{\mathrm{rep}}

\DeclareMathOperator{\E}{E}
\DeclareMathOperator{\Var}{Var}
\DeclareMathOperator{\var}{var}
\DeclareMathOperator{\Sd}{Sd}
\DeclareMathOperator{\sd}{sd}
\DeclareMathOperator{\Bin}{Bin}
\DeclareMathOperator{\Beta}{Beta}
\DeclareMathOperator{\Invchi2}{Inv-\chi^2}
\DeclareMathOperator{\NInvchi2}{N-Inv-\chi^2}
\DeclareMathOperator{\logit}{logit}
\DeclareMathOperator{\N}{N}
\DeclareMathOperator{\U}{U}
\DeclareMathOperator{\tr}{tr}
%\DeclareMathOperator{\Pr}{Pr}
\DeclareMathOperator{\trace}{trace}
\DeclareMathOperator{\rep}{\mathrm{rep}}

\pagestyle{empty}

\begin{document}
\thispagestyle{empty}

\section*{Bayesian data analysis -- reading instructions 6} 
\smallskip
{\bf Aki Vehtari}
\smallskip

\subsection*{Chapter 6}

Outline of the chapter 6
\begin{list}{$\bullet$}{\parsep=0pt\itemsep=2pt}
\item 6.1 The place of model checking in applied Bayesian statistics
\item 6.2 Do the inferences from the model make sense?
\item 6.3 Posterior predictive checking {\color{gray}($p$-values can be skipped)}
\item 6.4 Graphical posterior predictive checks (this can be skimmed, see instead the paper \textit{Visualization in Bayesian workflow})
\item 6.5 Model checking for the educational testing example
\end{list}

\noindent
R and Python demos at \url{https://avehtari.github.io/BDA_course_Aalto/demos.html}
\begin{list}{$\bullet$}{\parsep=0pt\itemsep=2pt}
\item demo6\_1: Posterior predictive checking - light speed
\item demo6\_2: Posterior predictive checking - sequential dependence
\item demo6\_3: Posterior predictive checking - poor test statistic
\item demo6\_4: Posterior predictive checking - marginal predictive p-value
\end{list}

\noindent
Find all the terms and symbols listed below. When reading the chapter,
write down questions related to things unclear for you or things you
think might be unclear for others.
\begin{list}{$\bullet$}{\parsep=0pt\itemsep=2pt}
\item model checking
\item sensitivity analysis
\item external validation
\item posterior predictive checking
\item joint posterior predictive distribution
\item marginal (posterior) predictive distribution
\item self-consistency check
\item replicated data
\item $y^{\rep}$, $\tilde{y}$, $\tilde{x}$
\item test quantities
\item discrepancy measure
\item tail-area probabilities
\item classical $p$-value
\item posterior predictive $p$-values
\item multiple comparisons
\item marginal predictive checks
\item cross-validation predictive distributions
\item conditional predictive ordinate
\end{list}

 \subsection*{Replicates vs. future observation}

 Predictive $\tilde{y}$ is the next not yet observed possible
 observation. $y^{\mathrm{rep}}$ refers to replicating the whole
 experiment (with same values of $x$) and obtaining as many replicated
 observations as in the original data.

 \subsection*{Posterior predictive $p$-values}

 Section 6.3 discusses posterior predictive $p$-values, which we don't
 recommend any more especially in a form of hypothesis testing.

 \subsection*{Prior predictive checking}

 Prior predictive checking using just the prior predictive
 distributions is very useful tool for assessing the sensibility of
 the model and priors even before observing any data or before doing
 the posterior inference. See additional reading below for examples.
 
 \subsection*{Additional reading}

The following article has some useful discussion and examples also about prior and posterior predictive checking.
\begin{itemize}
\item  Gabry, Simpson, Vehtari, Betancourt, and Gelman (2018). Visualization in Bayesian workflow. {\em Journal of the Royal Statistical Society Series A}, , 182(2):389-402. \url{https://doi.org/10.1111/rssa.12378}.
\item Video of the paper presentation \url{https://www.youtube.com/watch?v=E8vdXoJId8M}
\end{itemize}

And some useful demos
\begin{itemize}
  \item Graphical posterior predictive checks using the bayesplot package\\
    \url{http://mc-stan.org/bayesplot/articles/graphical-ppcs.html}
  \item Another demo \href{http://avehtari.github.io/BDA_R_demos/demos_rstan/ppc/poisson-ppc.html}{demos\_rstan/ppc/poisson-ppc.Rmd}
  \end{itemize}


\end{document}


%%% Local Variables: 
%%% TeX-PDF-mode: t
%%% TeX-master: t
%%% End: 
